\chapter{使用指南}\label{user_guide}


\section{如何開始撰寫自己的論文內容}
目前的文件數量仍然會讓您在評估上保有許多疑惑,為了避免讓您在撰寫論文時遇到許多的小問題,本章將會一步步帶著您將本專案修改為您自己的論文。

\section{架構簡介}
本專案所有的設定都盡可能的模組化,讓每個目錄、檔案的操作內容皆能專注在特定的事務上。在開始之前先一一介紹本專案的架構。

\begin{itemize}
    \item Chapter - 論文各章節文件
    \item Configurations - 論文設定 (套件設定)
    \item Externals - 外部文件匯入
    \item Figures - 圖片
    \item Fonts - 字體檔案
    \item Instance - 論文文章以外的文件 (封面、中英文摘要等..)
    \item Packages - LaTeX package
    \item References - 參考文獻
    \item Tables - 表格
    \item Templates - 版型 sty 檔案
\end{itemize}

\section{如何編輯}

\subsection*{Chapter 的新增、刪除、修改}

