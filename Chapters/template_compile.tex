
\chapter{使用此模版並編譯論文}\label{Compiled}

本模版使用 \href{https://tug.org/texlive/}{texlive} 編譯,並支援 \href{https://www.docker.com/}{Docker} 之套件 \href{https://hub.docker.com/r/texlive/texlive}{texlive/texlive} 進行 \LaTeX 論文編譯。

\section{Docker}\label{docker}

\href{https://www.docker.com/}{Docker} 是一個開放原始碼平台,用於建置、部署及管理容器化應用程式。瞭解容器、容器與 VM 的比較,以及為何 Docker 如此被廣泛地採用及使用。

\subsection{texlive}

使用 \href{https://hub.docker.com/r/texlive/texlive}{texlive/texlive} 套件,安裝方式如下

\begin{lstlisting}
    docker pull texlive/texlive // 下載 texlive 套件

\end{lstlisting}

\section{texlive}
\href{https://www.tug.org/texlive/}{Texlive}
這邊都是 Texlive 各國的鏡像下載區,台灣的 latex 鏡像站由元智大學及中正大學所建置,但因為中正大學的鏡像站已經Not Found 404,元智大學的鏡像站檔案下載的時候會 loss 掉很多檔案,最後我從台灣的鄰近國家下載 \href{http://ftp.kddilabs.jp/}{株式会社KDDI総合研究所}。

\href{https://texwiki.texjp.org/}{日本 latex wiki}

\section{MiKTeX}
經測試 \href{https://miktex.org/}{MiKTeX} 安裝完成能夠使用 make 指令編譯本模版,但使用 VSCODE 套件 LaTeX Workshop 編譯時,暫有些問題。

\section{編輯器}\label{editor}

推薦使用 \href{https://code.visualstudio.com/}{Visual Studio Code} (以下簡稱 VSCODE),在 VSCODE 中可利用多種套件使工作更便利。

\subsection{REMOTE}

VSCODE 的 \href{https://marketplace.visualstudio.com/items?itemName=ms-vscode-remote.vscode-remote-extensionpack}{Remote Development} 套件包允許您打開容器、遠端電腦或 Windows Linux 子系統 (WSL) 並利用 VSCODE 的完整功能集。 由於這使您可以在任何地方建立一個全職的開發環境,您可以:

\begin{itemize}
    \item    在您部署到的相同操作系統上進行開發,或者使用比本地電腦更大、更快或更專業的硬體。
    \item    在不同的獨立開發環境之間快速切換並進行更新,而不必擔心影響您的本地機器。
    \item    通過輕鬆啟動、一致的開發容器幫助新的團隊成員/貢獻者快速獲得生產力。
    \item    從功能齊全的開發工具中直接從舒適的 Windows 中利用基於 Linux 的工具鏈。
\end{itemize}

由於遠端開發直接在遠端電腦上運行命令和擴充,因此無需在本地機器上放置源代碼即可獲得這些好處。 


\subsection{LATEX}\label{latex_workshop}
\href{https://marketplace.visualstudio.com/items?itemName=James-Yu.latex-workshop}{LaTeX Workshop} 是 Visual Studio Code 套件,旨在為使用 Visual Studio Code 進行 LaTeX 排版提供核心功能。 

\subsection{PDF}

\href{https://marketplace.visualstudio.com/items?itemName=tomoki1207.pdf}{vscode-pdf} 為在 VSCODE 直接開啟 PDF,方便 LATEX 編譯完成之 PDF 檔案直接校閱。

\subsection{GIT}

\href{https://git-scm.com/}{git} 是一個開源的分佈式版本控制系統,用於敏捷高效地處理任何或小或大的項目。

Git 是 Linus Torvalds 為了幫助管理 Linux 核心開發而開發的一個開放源碼的版本控制軟體。

Git 與常用的版本控制工具 CVS, Subversion 等不同,它採用了分佈式版本庫的方式,不必服務器端軟件支持。

幫助論文進行版本控制。

\subsection{make}

欲使用 makefile 編譯此模版,在專案目錄底下新增 makefile 及 main/ 目錄底下新增 makefile。

\newpage

此檔案為專案根目錄底下 makefile

%[language=makefile]
\begin{lstlisting} 
    # file:  ./makefile
    # Date:  2021/10/15
    # Name:  YingChao
    
    
    # 主要文件名稱 (不需加副檔名)
    MAINDOC  = main
    
    # 是否使用預編譯
    precompiled = 0
    
    # 編譯文件放置目錄檔案
    BUILD_DIR = build
    
    # 設定目標及根目錄
    OUTPDF = main
    TARGET = ${BUILD_DIR}/${OUTPDF}.xdv
    
    # 讀取子目錄
    SUBDIRS = $(shell ls -l | grep ^d | awk '{if($$9 == "$(MAINDOC)") print $$9}')
    OUTDIRS = $(shell ls -l | grep ^d | awk '{if($$9 == "$(BUILD_DIR)") print $$9}')
    
    # 讀取 pdf 以外的輸出文件
    OUTFILE=$(shell ls -l $(OUTDIRS) | grep ^- | awk '{if($$9 != "$(MAINDOC).pdf") print $$9}')
    
    # 記住當前根目錄路徑
    ROOT_DIR=$(shell pwd)
    
    # commands to compile document
    MAKE	    = +make
    LATEX       = xelatex
    MAKEPDF	    = xdvipdfmx
    BIBTEX	    = biber
    GHOSTSCRIPT = ghostscript
    xDVI_PDFMx  = -E -o $(OUTPDF).pdf
    
    # commands parameter
    FILESYNC = -synctex=1
    FILE_ERR = -file-line-error
    INTERACT = -interaction=nonstopmode  #batchmode
    HALT_ERR = -halt-on-error
    NONE_PDF = -no-pdf
    OUTP_DIR = -output-directory=$(ROOT_DIR)/$(BUILD_DIR)
    OUTP_DRV = -output-driver='xdvipdfmx -z0' # 此參數設定 PDF 壓縮等級
    
    ifeq ($(precompiled), 1)
    PRECOMPL = -shell-escape "&$(LATEX)"
    
    TEX_Parameter = $(NONE_PDF) $(OUTP_DIR) $(OUTP_DRV) \
                    $(HALT_ERR) $(PRECOMPL) 
    else
    TEX_Parameter = $(FILESYNC) $(FILE_ERR) $(INTERACT) \
                    $(HALT_ERR) $(NONE_PDF) $(OUTP_DIR) \
                    $(OUTP_DRV)
    endif
    
    # 宣告以下為廣域變數,使子目錄的 makefile 也能讀取以下變數
    export MAINDOC ROOT_DIR BUILD_DIR OUTP_DIR
    export LATEX BIBTEX TEX_Parameter precompiled
    export MAKE MAKEPDF xDVI_PDFMx OUTPDF
    
    NoPrintDir = --no-print-directory # make 不顯示進入/離開目錄訊息
    
    all: ${TARGET}
    
    ${TARGET}:
        @echo "\033[33m make a directory \033[0m"
        @mkdir -p $(BUILD_DIR)
        @$(MAKE) -C $(SUBDIRS) $(NoPrintDir)
    
    clean:
        @rm -rf $(OUTDIRS)
    
    staypdf:
    ifeq ($(strip $(OUTFILE)),)
        @echo "\033[1;31m [MSG] file already cleaned \033[0m"
    else
        @echo "\033[1;31m [MSG] start clean file\033[0m"
        @cd $(OUTDIRS);\
        rm $(OUTFILE)
        @echo "\033[1;32m [MSG] Retain pdf , other file for deleting \033[0m"
    endif
    
    
    # https://www.internalpointers.com/post/compress-pdf-file-ghostscript-linux
    # sudo apt install ghostscript
    
    # 		-sOwnerPassword=mypassword \
    
    compress:
        $(GHOSTSCRIPT) -sDEVICE=pdfwrite \
        -dCompatibilityLevel=1.4 \
        -dPDFSETTINGS=/printer \
        -dQUIET -dBATCH -dNOPAUSE	\
        -dEmbedAllFonts=false \
          -dSubsetFonts=true \
          -dConvertCMYKImagesToRGB=true \
          -dCompressFonts=true \
        -sOutputFile=${BUILD_DIR}/$(MAINDOC)_out.pdf \
        ${BUILD_DIR}/$(MAINDOC).pdf
        @echo -e "\033[32m [MSG] 		PDF Size is compressed\033[0m"	
        @cd $(BUILD_DIR);\
        rm -f *.xdv;\
        rm -f $(MAINDOC).pdf;\
        mv $(MAINDOC)_out.pdf $(MAINDOC).pdf
    
    
    .PHONY: clean all
    
 \end{lstlisting}

 \newpage

 此檔案為 main 目錄底下 makefile

%[language=makefile]
 \begin{lstlisting}
    # file: main/makefile
    # Date:  2022/06/11
    # Name:  YingChao
    
    # source files
    TEX_FILES=${wildcard *.tex}
    XDV_FILES = $(ROOT_DIR)/$(BUILD_DIR)/${MAINDOC}.xdv
    
    CURRENTDIR=$(shell pwd)
    
    
    all:
        @echo "\033[33m $(shell pwd) \033[0m"
    ifeq ($(precompiled), 1)
        @$(MAKE) xeINItex				# initialize
        @echo "\033[32m [MSG] 		Initialized \033[0m"
    endif
        @$(MAKE) xetex					# 1st commpiling document
        @echo "\033[32m [MSG] 		1st compiled \033[0m"
        @$(MAKE) bibtex					# reference commpiling
        @echo "\033[32m [MSG] 		Reference compiled  \033[0m"
        @$(MAKE) xetex					# 2nd commpiling document
        @echo "\033[32m [MSG] 		2nd compiled \033[0m"
        @$(MAKE) xetex					# 3rd commpiling document 
        @echo "\033[32m [MSG] 		3rd compiled \033[0m"
        @$(MAKE) xDVIPDFMx				# Genrated PDF
        @echo "\033[32m [MSG] 		Successly compiled  \033[0m"
    
    xetex:
        $(LATEX) $(TEX_Parameter) $(TEX_FILES)
    
    bibtex:
        $(BIBTEX) $(ROOT_DIR)/$(BUILD_DIR)/$(MAINDOC)
    
    xDVIPDFMx:
        @cd $(ROOT_DIR)/$(BUILD_DIR);\
        $(MAKEPDF) -vv $(xDVI_PDFMx) $(XDV_FILES)
    
    xeINItex:
        $(LATEX) $(OUTP_DIR) -initialize -jobname="$(MAINDOC)" "&$(LATEX)" mylatexformat.ltx """$(MAINDOC).tex""" 
    
    # https://ctan.math.utah.edu/ctan/tex-archive/macros/latex/contrib/mylatexformat/mylatexformat.pdf
    
 \end{lstlisting}
 

透過 make 命令執行論文 PDF 的編譯,或是透過 \ref*{latex_workshop} 套件的功能編譯本論文模版。

\begin{description}
    \item[make all] 編譯論文 PDF ,也可直接使用 make 命令。
    \item[make clean] 清除整個 build 目錄。
    \item[make staypdf] 清除 build 目錄底下 PDF 以外的檔案。
\end{description}

\subsection{cmake}

動作順序為

\begin{description}
    \item[mkdir build] 建立 build 目錄
    \item[cd build]  進入 build 目錄
    \item[cmake ..]  cmake 建立編譯專案生成 makefile
    \item[make]      make 命令編譯論文專案
\end{description}


也可以是 \newline
cmake -S . -B build \newline
cmake --build build \newline