\chapter{緒論}\label{explanation}


\section{前言}\label{1-1}
想在畢業之前多學習一些東西,除了常用 office 軟體之外,更想挑戰自我使用 \LaTeX 撰寫畢業論文\cite{learn-latex}。
透過 F Mittelbach 所撰寫的 LaTeX 夥伴一文中 \cite{mittelbach2004} 可以學習到 LaTeX\cite{the-latex-project} 的使用方式,其實最大的門檻還是起步的樣板(模板),在朝陽科大碩士學位論文研究 \cite{ShellLATEX} 中,使用 Shell 的方式;與本樣板使用 makefile 方式大同小異。

\subsection{為什麼使用 LaTeX?}
使用 Microsoft Office Word 寫論文也可以完成學位呀!為什麼要這麼麻煩?特地使用一個自己不熟悉的軟體進行論文撰寫呢?
難到使用 \LaTeX 寫出來的論文就會比較高級嗎?

使用自己熟悉的軟體撰寫文件哪有什麼困難的?你若一直在舒適圈不願意出來看看外面的世界,那你這個即將畢業的學生就有半退休心態的公務員有什麼兩樣呢?
據說使用 \LaTeX 在 Linux 上編譯出來的 PDF 檔,是無法逆向回去到 Office Word 格式的,尚未驗證過無法保證,倘若這謠傳是真實的,更能保證論文不被有心人士修改/盜用。

\clearpage

\hbox{使用 LaTeX 撰寫論文,將好處整理並羅列如下:}

\begin{itemize}
\item 將內容變成純文字檔,可以更方便透過 git \cite{git-version-control} \cite{git20190914} 對論文作控管版本 \cite{spinellis2012git},在與教授進行論文修改的同時可以將之前所撰寫的紀錄保存起來 \cite{chacon2014pro}。
\item 可以將怒氣發洩在註解上(誤),能夠透過註解有效紀錄 Todo 事項。
\item 與寫程式使用相同的 IDE , 能在同一個 IDE 下完成許多事情。
\item 不用在意標點符號溢出邊界問題,不用在意交互參照問題,在此樣板中已設定完成。只需專心撰寫論文內容後,在 Terminal 下 Command 就完成 pdf 檔案。
\item 整份文件的一致性,讓您的文件看起來更專業。
\item 若有想到新的優點再更新上來。
\end{itemize}

\newpage

\section{製作樣板動機}\label{1-2}
使用 \LaTeX 的過程中,最困難的那一步並不是架設環境,也不是建立文字內容,是文件的版面設定(邊框多少、浮水印的透明度、套件衝突等...),為幫助有心想使用 LaTeX 卻沒有時間設定瑣碎事務的人們。
也建立自己在畢業前對學校的貢獻度(或許南台根本不在意?),也可能畢業十年、廿十年都不會被發現這個放在 github 上設為 pulic 的 repository。若學校有看到希望能加入到教務處的論文樣板中,讓更多學生能夠看見這份為南臺精心設計的 latex 論文樣板。

\section{樣板架構}\label{1-3}
\n 本論文樣板編排方式如下:\\
第\ref{explanation}章 說明本製作樣板動機與好處\\
第\ref{example}章 本樣板精心製作 \LaTeX 語法範例\\
第\ref{algorithm}章 演算法及程式碼插入語法範例\\
第\ref{result}章 驗證樣板結果\\
第\ref{conclusion}章 樣板設計結論與對此樣本的未來期許
