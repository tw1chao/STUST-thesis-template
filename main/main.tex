%========================================================================================
% STUST MEng. Thesis
% Original authors : AIoRLab/NCLab/WNDCLab Thesis LaTeX Team
% Department of Electronic Engineering
% National Souther Taiwan of Sciences and Technology
% GitHub : https://github.com/yingchao-chen/STUST-thesis-template
% Version 1.01 (2020.03.14)
%========================================================================================

%----------------------------------------------------------------------------------------
%	PACKAGES AND OTHER DOCUMENT CONFIGURATIONS
%----------------------------------------------------------------------------------------

\documentclass[oneside,
12pt, % The default document font size, options: 10pt, 11pt, 12pt
english, % ngerman for German
onehalfspacing, % Single line spacing, alternatives: onehalfspacing or doublespacing
]{../Configurations/STUST-MS-Thesis} % The class file specifying the document structure

\setcounter{secnumdepth}{6}
\setcounter{tocdepth}{6}

\usepackage{xunicode}
\usepackage{mathptmx}

%----------------------------------------------------------------------------------------
% 更變章節目錄的套件
%----------------------------------------------------------------------------------------
\usepackage{titlesec, titletoc}
\titlecontents{chapter}[0em]{}
{\hspace{0em}\makebox[4.1em][l]
	{第\zhnumber{\thecontentslabel}章}}{}
{\titlerule*[0.7pc]{.}\contentspage}

\titleformat{\chapter}{\centering\Huge\bfseries}{Chapter\,{\thechapter}}{1em}{}
\titleformat{\chapter}{\centering\Huge\bfseries}{第\,\zhnumber{\thechapter}\,章}{1em}{}    %設定章節樣式

%----------------------------------------------------------------------------------------
% 中国語,日本語,韓国語使用の設定
%----------------------------------------------------------------------------------------
\def\fontpath{Path=\string../fonts/}
\usepackage{lmodern}
\usepackage[BoldFont,SlantFont]{xeCJK}
\xeCJKsetup{PunctStyle=kaiming}
\usepackage{lipsum}
\usepackage{fontspec}
% \usepackage{type1cm,type1ec} % If your installation uses scaleable versions of the Computer Modern or European Computer Modern (EC) fonts
\defaultfontfeatures{AutoFakeBold=2.5,AutoFakeSlant=.2}
\setsansfont[\fontpath]{times}
\setmainfont[
	\fontpath,
	Extension      = .ttf,
	BoldFont       = *bd,
	ItalicFont     = *i,
	BoldItalicFont = *bi
	]{times}        %預設英文字體

\defaultCJKfontfeatures{AutoFakeBold=2.5,AutoFakeSlant=.2}
\CJKfontspec[\fontpath]{ukai}
\setCJKmainfont[\fontpath]{ukai.ttc}
\newCJKfontfamily\kai{ukai}[\fontpath,AutoFakeBold]
\setCJKmonofont[\fontpath]{edukai-4.0}
\setCJKsansfont[\fontpath]{ukai}
\setCJKfamilyfont{edukai-4.0}[\fontpath]{edukai-4.0}       %教育部楷書
\newcommand*{\edukai}{\CJKfamily{edukai-4.0}}


% 解決 XeTeX 中文的斷行問題
\XeTeXlinebreaklocale "zh"
\XeTeXlinebreakskip = 0pt plus 1pt
\XeTeXlinebreakskip = 0pt plus 1pt minus 0.1pt

%----------------------------------------------------------------------------------------
\usepackage{array}
\usepackage{longtable}
\usepackage{colortbl}			%表格標題註解之巨集套件

\usepackage[final]{pdfpages}    %PDFの各ページを挿入するためのパッケージです

\usepackage{url}         %URLをリンクとして表示するためのパッケージ
\usepackage{zhnumber}
\linespread{1.5} %行距1.5倍

\usepackage{docmute}
% \usepackage[utf8]{inputenc} % Required for inputting international characters
\usepackage[LGR,OT1]{fontenc} % Output font encoding for international characters
\usepackage{palatino} % Use the Palatino font by default
\usepackage{indentfirst} \setlength{\parindent}{2em} %首段文章縮排套件
\usepackage{xcolor}
\usepackage{lettrine}  % used for chinese bigger capital
\usepackage{../Package/slashbox}
\usepackage{booktabs}
\usepackage{enumerate}
\usepackage{multirow}
\usepackage{forest}
\usepackage{ifthen}
\usepackage[strict]{changepage}

%----------------------------------------------------------------------------------------
% 參考文獻 Reference package
%----------------------------------------------------------------------------------------
\usepackage[redeflists]{IEEEtrantools}   %複数行にわたった数式を書く場合には,IEEEeqnarray 環境が便利です
\usepackage[hyperref=true,backend=biber,sorting=none,backref=true]{biblatex}
\addbibresource{\RefPath} % The filename of the bibliography

\frontmatter % Use roman page numbering style (i, ii, iii, iv...) for the pre-content pages
\pagestyle{plain} % Default to the plain heading style until the thesis style is called for the body content
\usepackage{makecell}
\cleardoublepage
\usepackage[autostyle=true]{csquotes} % Required to generate language-dependent quotes in the bibliography
\usepackage{datetime}
\usepackage{xparse}

%----------------------------------------------------------------------------------------
% 繪圖函數 Graph functions
%----------------------------------------------------------------------------------------
\usepackage{pgfplots}
\usepgfplotslibrary{colorbrewer}
\pgfplotsset{compat=newest,compat/show suggested version=false}

%----------------------------------------------------------------------------------------
% 圖片套件 Image functions
%----------------------------------------------------------------------------------------
\usepackage{tikz} % Required for drawing custom shapes
\usepackage{eso-pic,picture} % \AddToShipoutPictureBG is esp-pic package's function

%----------------------------------------------------------------------------------------
%	Source Code include Setup
%----------------------------------------------------------------------------------------
\usepackage{listings}
\lstset{
			language={C},   %言語の指定.C言語ならCとします
			basicstyle={\ttfamily}, %標準の書体
			identifierstyle={\small},
			commentstyle={\small\ttfamily \color[rgb]{0,0.5,0}},    %注釈の書体 
			keywordstyle={\small\bfseries \color[rgb]{0,0,1}},      %キーワード(int, ifなど)の書体
			stringstyle={\small\ttfamily \color[rgb]{1,0,1}},
			frame=tRBl, %フレームの指定
			framesep=10pt, %フレームと中身(コード)の間隔
			breaklines=true, %行が長くなったときの自動改行
			linewidth=15cm, %フレームの横幅
			lineskip=-0.5ex, %行間の調整
			columns=[l]{fullflexible},  %書体による列幅の違いを調整するか
			numbers=left,
			stepnumber=1,   %行番号をいくつとばしで表示するか
			numbersep=14pt,
			tabsize=2, %Tabを何文字幅にするかの指定
			morecomment=[l]{//}
		}

%
% Thesis Defined
%

\ifx\CoAdvisor\undefined
\newif\ifCoAdvisor  % 是否有共同指導教授參與


% 有共同指導教授
% \CoAdvisortrue
% 沒有共同指導教授
\CoAdvisorfalse


% 參考文獻[BIB]檔名
\def\RefName{reference}


% 研究生中文名字
\def\authortwname{君の名は}
% 研究生英文名字		
\def\authorenname{YourName}		


% 指導教授中文名字
\def\Advisortwname{林默娘}  
% 指導教授英文名字
\def\Advisorenname{Mo-Niang Lin Ph.D.}  


\ifCoAdvisor
% 共同指導教授中文名字
\def\CoAdvisortwname{弁慶}
% 共同指導教授英文名字
\def\CoAdvisorenname{Benkei Ph.D.}  
\fi

\def\schooltwname{南臺科技大學}
\def\schoolenname{Souther Taiwan University of Sciences and Technology}
\def\schoolenlocation{Tainan, Taiwan, Republic of China}

\def\majortwname{Electronic Engineering}  % 主修項目
\def\depttwname{電子工程系}  % 系所中文名稱
\def\deptenname{Department of Electronic Engineering}  % 系所英文名稱

\def\degreetw{碩士}  % 學位
\def\degreeen{Master}  % 學位

\def\titletw{南臺科技大學 \LaTeX 論文樣板}  % 論文題目(中文)
\def\titleen{STUST \LaTeX \ Thesis Template}  % 論文題目(英文)

\def\dateROC{中華民國 一〇九 年 六 月}  % 日期(月份)
\def\dateen{Jun, 2020}  % 英文日期

\def\thesistitlelogo{Figures/Logos/stust.png}  % 主頁面logo
\def\watermarkimage{Figures/Logos/stust.png}  % 浮水印


\draftfalse         % 未完稿   (drafttrue 完稿)

\begin{document}
\newcommand\n{\mbox{\qquad}}

\frontmatter % Use roman page numbering style (i, ii, iii, iv...) for the pre-content pages
\pagestyle{plain} % Default to the plain heading style until the thesis style is called for the body content

%----------------------------------------------------------------------------------------
%	論文編印項目次序
%   編排順序依照學術論文格式規範文件進行排序
%----------------------------------------------------------------------------------------

% ----- 封面頁 -----
%----------------------------------------------------------------------------------------
%	Front Cover 封面
%----------------------------------------------------------------------------------------

\begin{titlepage}
\vspace*{2.1em}

\begin{center}

{\fontsize{36pt}{12pt}\bfseries{\edukai \schooltwname}\\
\vspace{8mm}
\fontsize{24pt}{12pt}\bfseries{\edukai \depttwname{\degreetw 班}}\\
\vspace{5mm}
\fontsize{24pt}{12pt}\bfseries{\edukai \degreetw 學位論文}\\
\vspace{30mm}}

{\fontsize{24pt}{12pt}\textbf{\edukai \titletw}}
\vspace{3mm}

% \vspace{1\baselineskip}
{\fontsize{22pt}{0pt}\textbf{\titleen}}\\

\ifdraft
\vspace{25mm}
\else
\textcolor{blue}{\Large{\today(\edukai \version)}}
\vspace{20mm}
\fi

\vspace{3.5\baselineskip}


% \begin{flushleft}
% \begin{adjustwidth}{10em}{}
\begin{tabular}{rl}
\fontsize{18pt}{0pt}\selectfont{\makebox[4em][s]{\textbf{\edukai 研\hspace{\fill}究\hspace{\fill}生}} %
\textbf{:}} &
\fontsize{18pt}{0pt}\selectfont{\makebox[4em][l]{\textbf{\edukai \authortwname}}} \\[20mm]

\fontsize{18pt}{0pt}\selectfont{\makebox[4em][s]{\textbf{\edukai 指\hspace{\fill}導\hspace{\fill}教\hspace{\fill}授}} %
\textbf{:}} &
\fontsize{18pt}{0pt}\selectfont{\makebox[4em][l]{\textbf{\edukai \Advisortwname}}} \\

\ifCoAdvisor
\fontsize{18pt}{0pt}\selectfont{\makebox[4em][s]{\textbf{\edukai 共\hspace{\fill}同\hspace{\fill}指\hspace{\fill}導}} %
\textbf{:}} &
\fontsize{18pt}{0pt}\selectfont{\makebox[4em][l]{\textbf{\edukai \CoAdvisortwname}}} \\[8mm]
\else
\vspace{8mm}
\fi
\end{tabular}
% \end{adjustwidth}
% \end{flushleft}

\vspace{7mm}
\fontsize{18pt}{0pt}{\bfseries{\edukai \dateROC }}
\end{center}

\end{titlepage} 
\clearpage


% ----- 書  脊 -----
% \clearpage

\rotatebox{90}{test測試}

\clearpage

% ----- 空白頁 -----
\ifdraft
\newpage \thispagestyle{empty}\newpage
\fi

% ----- 書名頁 -----
% %----------------------------------------------------------------------------------------
% %	書名頁
% %----------------------------------------------------------------------------------------

%----------------------------------------------------------------------------------------
%	Front Cover 封面
%----------------------------------------------------------------------------------------

\begin{titlepage}
\vspace*{1mm}

\begin{center}

{\fontsize{36pt}{10pt}\bfseries{\textbf{\schooltwname}}\\
\vspace{15mm}
\fontsize{24pt}{10pt}\bfseries{\depttwname{\textbf{\degreetw 班}}}\\
\vspace{5mm}
\fontsize{24pt}{10pt}\bfseries{\textbf{\degreetw 學位論文}}\\
\vspace{30mm}}

{\fontsize{24pt}{10pt} \textbf{\titletw}}
\vspace{5mm}

% \vspace{1\baselineskip}
{\fontsize{22pt}{10pt}  \textbf{\titleen}}\\
%\LARGE  {(初稿)}
\vspace{30mm}

\vspace{0.5\baselineskip}

\fontsize{18pt}{10pt} \authortwname
\vspace{20mm}

\begin{tabular}{rl}
\fontsize{18pt}{10pt}\makebox[6em][s]{指\hspace{\fill}導\hspace{\fill}教\hspace{\fill}授}: &
\fontsize{18pt}{10pt} \Advisortwname \\[5mm]

\ifCoAdvisor
\fontsize{18pt}{10pt}\makebox[6em][s]{共\hspace{\fill}同\hspace{\fill}指\hspace{\fill}導}: &
\fontsize{18pt}{10pt} \CoAdvisortwname \\[8mm]
\else
\vspace{8mm}
\fi
\end{tabular}

\vspace{15mm}
{\fontsize{18pt}{10pt} \textsc{ \dateROC }}
\end{center}

\end{titlepage} 
\clearpage

% \begin{titlepage}
% \vspace*{1mm}

% \begin{center}

% {\LARGE\bfseries  \titletw}
% \vspace{15mm}

% {\LARGE  \titleen}
% \vspace{15mm}

% {\large\bfseries{研究生:}\large\authortwname\\
% \large\bfseries{指導教授:}\large\Advisortwname\\
% \large\bfseries{共同指導:}\large\CoAdvisortwname}

% \vspace{15mm}
% {\Large\bfseries{\schooltwname}\\
% \vspace{4.5mm}
% \Large\bfseries{\depttwname{\degreetw 班}}\\
% \vspace{4.5mm}
% \Large\bfseries \degreetw 論文}\\
% \vspace{10mm}

% \
% \vspace{4.5mm}
% A Thesis Submitted to \deptenname\\
% \schoolenname\\
% in Partial Fulfillment of the Requirements\\
% for the Degree of \degreeen of Engineering\\
% in \majortwname

% \vspace{15mm}
% \dateen\\
% \schoolenlocation

% % \vspace{10mm}
% % \schoolenoldname is the predecessor of\\
% % National Kaohsiung University\\
% % of Science and Technology (renamed on Feb. 1, 2018)

% \vspace{10mm}
% {\large\bfseries{\dateROC}}

% \end{center}

% \end{titlepage} 


% ----- 博碩士論文授權書 -----
% \includepdf[pagecommand={\thispagestyle{empty}}]{Externals/powerofattorney.pdf}

% ----- 博士論文指導教授推薦書 (碩士可直接註解省略) -----
% \includepdf[pagecommand={\thispagestyle{empty}}]{Externals/recommend.pdf}

% 論文口試委員會審定書
% \includepdf[pagecommand={\thispagestyle{empty}}]{Externals/sign.pdf}

%----------------------------------------------------------------------------------------
%	論文內容 由此開始 
%----------------------------------------------------------------------------------------
\phantomsection     % 加這個命令後,目錄中的超鏈接才指向正確的頁碼
\STUSTwatermark     % 浮水印設定
\cleardoublepage    % 加這個命令後,目錄中的頁碼才顯示正確
\setcounter{page}{1}        % 摘要為第一頁
\pagenumbering{roman}       % 用羅馬RS編號

% 中英文摘要
\renewcommand{\abstractname}{摘\hspace{3em}要}
\clearpage
\phantomsection
\begin{cntabstract}

\n 隨著目前科技越來越進步,也使得人們的生活越來越便捷... 剩下的 交給你了!

\hbox{}
\it{關鍵詞:人工智慧、物聯網}
\end{cntabstract}
 \addchaptertocentry{\abstractname}
\renewcommand{\abstractname}{Abstract}
\clearpage
\phantomsection
\begin{engabstract} 

\n	This project is a template of the master's thesis \LaTeX of STUST , and it is arranged in accordance with the specifications of STUST's degree thesis format and the format of the electronic engineering department.

	\hbox{}
	\it{Keywords: STUST, Master's Thesis, Template}

\end{engabstract} \addchaptertocentry{\abstractname}

% 誌謝 or 序言 (視需要)
\renewcommand{\acknowledgementname}{誌\hspace{2em}謝}
\clearpage
\phantomsection
\begin{acknowledgements}

感恩 SeaFood !讚嘆 SeaFood !讓我享受到 SeaFood 的美味 ~活著真好!下次再一起去看日出吧


\end{acknowledgements} 
 \addchaptertocentry{\acknowledgementname}

% 目次
\phantomsection
\renewcommand{\contentsname}{\edukai 目\hspace{2em}次}
\cleardoublepage
\addchaptertocentry{\contentsname}
\tableofcontents
\cleardoublepage

% 表目錄
\phantomsection
\renewcommand{\listtablename}{\edukai 表\hspace{.5em}目\hspace{.5em}錄}
\newcommand{\lotlabel}{表}
\renewcommand{\numberline}[1]{\lotlabel~#1\hspace*{1em}}
\renewcommand{\tablename}{表}
\listoftables \addchaptertocentry{\listtablename}
\newpage

% 圖目錄
\phantomsection
\newcommand{\loflabel}{圖}
\renewcommand{\listfigurename}{\edukai 圖\hspace{.5em}目\hspace{.5em}錄}
\renewcommand{\figurename}{圖}
\renewcommand{\numberline}[1]{\loflabel~#1\hspace*{1em}}
\listoffigures \addchaptertocentry{\listfigurename}
\newpage

% 論文本文
\mainmatter % Begin numeric (1,2,3...) page numbering
% \ifpdf
%     \graphicspath{{MyFigures/chapter1/PNG/}{MyFigures/chapter1/PDF/}{MyFigures/chapter1/}}
% \else
%     \graphicspath{{MyFigures/chapter1/EPS/}{MyFigures/chapter1/}}
% \fi

\chapter{緒論}\label{1}


\section{前言}\label{1-1}
卒業して欲しい \cite{m1} ,てすとてすとてすとてすとてすとてすとてすとてすとてすとてすとてすとてすとてすとてすとてすとてすとてすとてすとてすとてすとてすとてすとてすとてすとてすとてすとてすとてすとてすとてすとてすと\cite{talbot97} 。

\subsection{てすと}


\subsection{テストテスト}
想不出來要寫什麼嗎?可能會寫不出來呢!\cite{goossens97}

\newpage

\section{研究動機}\label{1-2}
こんにちは、これはてすとだ!

\newpage

\section{論文架構}\label{1-3}

\n 本論文編排方式如下:

第\ref{2}章 說明本研究平台的硬體配備說明,並介紹系統原理與平台架構

第\ref{3}章 說明系統架構與操作

第\ref{4}章 驗證系統的結果
		
         驗證系統之結果

         驗證系統1與系統2整合之結果

第\ref{5}章 結論與未來展望

\chapter{實驗方法及裝置}


\section{架構}
為了防止世界被破壞$\sim$ \\
為了守護世界的和平$\sim$


\chapter{結果}\label{result}

\section{成功編譯}
透過 make 指令能夠協助您有效完成建立 PDF 檔案。

\clearpage



\chapter{結論與未來展望}\label{conclusion_and_future}

\section{研究結論}

嗯 結論

\section{未來展望}

很多人放不下過去,不是因為他們重回憶重感情,
而是他們不知道未來在哪裡!就像我一樣。


%----------------------------------------------------------------------------------------
% 參考文獻  文献データベース
\newpage

\renewcommand{\bibname}{參考文獻}

\urlstyle{IEEEtran}
\printbibliography


% 透過類型分類參考文獻
% \printbibliography[type=article,title={Articles only}]
% \printbibliography[type=book,title={Books only}]

% 透過關鍵字分類參考文獻
% \printbibliography[keyword={git}},title={git only}]
% \printbibliography[keyword={latex},title={\LaTeX-related only}]
 \addchaptertocentry{\bibname}
%----------------------------------------------------------------------------------------

% 附錄及符號(公式)彙編(視需要)
% \appendix
% \input{../Appendices/AppendixA}

%----------------------------------------------------------------------------------------
%	論文內容 結束
%----------------------------------------------------------------------------------------

% 自傳或簡歷 (可有可無)


%----------------------------------------------------------------------------------------

\clearpage
\end{document}
