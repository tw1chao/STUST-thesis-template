%----------------------------------------------------------------------------------------
%	PACKAGES AND OTHER DOCUMENT CONFIGURATIONS
%----------------------------------------------------------------------------------------

\documentclass[oneside,
12pt, % The default document font size, options: 10pt, 11pt, 12pt
english, % ngerman for German
onehalfspacing, % Single line spacing, alternatives: onehalfspacing or doublespacing
]{../Configurations/STUST-MS-Thesis} % The class file specifying the document structure

\setcounter{secnumdepth}{6}
\setcounter{tocdepth}{6}

\usepackage{xunicode}
\usepackage{mathptmx}

%----------------------------------------------------------------------------------------
% 更變章節目錄的套件
%----------------------------------------------------------------------------------------
\usepackage{titlesec, titletoc}
\titlecontents{chapter}[0em]{}
{\hspace{0em}\makebox[4.1em][l]
	{第\zhnumber{\thecontentslabel}章}}{}
{\titlerule*[0.7pc]{.}\contentspage}

\titleformat{\chapter}{\centering\Huge\bfseries}{Chapter\,{\thechapter}}{1em}{}
\titleformat{\chapter}{\centering\Huge\bfseries}{第\,\zhnumber{\thechapter}\,章}{1em}{}    %設定章節樣式

%----------------------------------------------------------------------------------------
% 中国語,日本語,韓国語使用の設定
%----------------------------------------------------------------------------------------
\def\fontpath{Path=\string../fonts/}
\usepackage{lmodern}
\usepackage[BoldFont,SlantFont]{xeCJK}
\xeCJKsetup{PunctStyle=kaiming}
\usepackage{lipsum}
\usepackage{fontspec}
% \usepackage{type1cm,type1ec} % If your installation uses scaleable versions of the Computer Modern or European Computer Modern (EC) fonts
\defaultfontfeatures{AutoFakeBold=2.5,AutoFakeSlant=.2}
\setsansfont[\fontpath]{times}
\setmainfont[
	\fontpath,
	Extension      = .ttf,
	BoldFont       = *bd,
	ItalicFont     = *i,
	BoldItalicFont = *bi
	]{times}        %預設英文字體

\defaultCJKfontfeatures{AutoFakeBold=2.5,AutoFakeSlant=.2}
\CJKfontspec[\fontpath]{ukai}
\setCJKmainfont[\fontpath]{ukai.ttc}
\newCJKfontfamily\kai{ukai}[\fontpath,AutoFakeBold]
\setCJKmonofont[\fontpath]{edukai-4.0}
\setCJKsansfont[\fontpath]{ukai}
\setCJKfamilyfont{edukai-4.0}[\fontpath]{edukai-4.0}       %教育部楷書
\newcommand*{\edukai}{\CJKfamily{edukai-4.0}}


% 解決 XeTeX 中文的斷行問題
\XeTeXlinebreaklocale "zh"
\XeTeXlinebreakskip = 0pt plus 1pt
\XeTeXlinebreakskip = 0pt plus 1pt minus 0.1pt

%----------------------------------------------------------------------------------------
\usepackage{array}
\usepackage{longtable}
\usepackage{colortbl}			%表格標題註解之巨集套件

\usepackage[final]{pdfpages}    %PDFの各ページを挿入するためのパッケージです

\usepackage{url}         %URLをリンクとして表示するためのパッケージ
\usepackage{zhnumber}
\linespread{1.5} %行距1.5倍

\usepackage{docmute}
% \usepackage[utf8]{inputenc} % Required for inputting international characters
\usepackage[LGR,OT1]{fontenc} % Output font encoding for international characters
\usepackage{palatino} % Use the Palatino font by default
\usepackage{indentfirst} \setlength{\parindent}{2em} %首段文章縮排套件
\usepackage{xcolor}
\usepackage{lettrine}  % used for chinese bigger capital
\usepackage{../Package/slashbox}
\usepackage{booktabs}
\usepackage{enumerate}
\usepackage{multirow}
\usepackage{forest}
\usepackage{ifthen}
\usepackage[strict]{changepage}

%----------------------------------------------------------------------------------------
% 參考文獻 Reference package
%----------------------------------------------------------------------------------------
\usepackage[redeflists]{IEEEtrantools}   %複数行にわたった数式を書く場合には,IEEEeqnarray 環境が便利です
\usepackage[hyperref=true,backend=biber,sorting=none,backref=true]{biblatex}
\addbibresource{\RefPath} % The filename of the bibliography

\frontmatter % Use roman page numbering style (i, ii, iii, iv...) for the pre-content pages
\pagestyle{plain} % Default to the plain heading style until the thesis style is called for the body content
\usepackage{makecell}
\cleardoublepage
\usepackage[autostyle=true]{csquotes} % Required to generate language-dependent quotes in the bibliography
\usepackage{datetime}
\usepackage{xparse}

%----------------------------------------------------------------------------------------
% 繪圖函數 Graph functions
%----------------------------------------------------------------------------------------
\usepackage{pgfplots}
\usepgfplotslibrary{colorbrewer}
\pgfplotsset{compat=newest,compat/show suggested version=false}

%----------------------------------------------------------------------------------------
% 圖片套件 Image functions
%----------------------------------------------------------------------------------------
\usepackage{tikz} % Required for drawing custom shapes
\usepackage{eso-pic,picture} % \AddToShipoutPictureBG is esp-pic package's function

%----------------------------------------------------------------------------------------
%	Source Code include Setup
%----------------------------------------------------------------------------------------
\usepackage{listings}
\lstset{
			language={C},   %言語の指定.C言語ならCとします
			basicstyle={\ttfamily}, %標準の書体
			identifierstyle={\small},
			commentstyle={\small\ttfamily \color[rgb]{0,0.5,0}},    %注釈の書体 
			keywordstyle={\small\bfseries \color[rgb]{0,0,1}},      %キーワード(int, ifなど)の書体
			stringstyle={\small\ttfamily \color[rgb]{1,0,1}},
			frame=tRBl, %フレームの指定
			framesep=10pt, %フレームと中身(コード)の間隔
			breaklines=true, %行が長くなったときの自動改行
			linewidth=15cm, %フレームの横幅
			lineskip=-0.5ex, %行間の調整
			columns=[l]{fullflexible},  %書体による列幅の違いを調整するか
			numbers=left,
			stepnumber=1,   %行番号をいくつとばしで表示するか
			numbersep=14pt,
			tabsize=2, %Tabを何文字幅にするかの指定
			morecomment=[l]{//}
		}
