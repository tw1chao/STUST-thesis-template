%----------------------------------------------------------------------------------------
%	PACKAGES AND OTHER DOCUMENT CONFIGURATIONS
%----------------------------------------------------------------------------------------

\documentclass[oneside,
12pt, % The default document font size, options: 10pt, 11pt, 12pt
english, % ngerman for German
onehalfspacing, % Single line spacing, alternatives: onehalfspacing or doublespacing
]{Configurations/STUST-MS-Thesis} % The class file specifying the document structure

\setcounter{secnumdepth}{6}
\setcounter{tocdepth}{6}
% \newtheorem{note}{註:}

\usepackage{ifthen}
\usepackage{xunicode}
\usepackage{mathptmx}
\usepackage{lmodern}

%----------------------------------------------------------------------------------------
%見た目を変更するパッケージとして
%----------------------------------------------------------------------------------------
\usepackage{titlesec, titletoc}   
\titlecontents{chapter}[0em]{}
{\hspace{0em}\makebox[4.1em][l]
	{第\zhnumber{\thecontentslabel}章}}{}
{\titlerule*[0.7pc]{.}\contentspage}

\titleformat{\chapter}{\centering\Huge\bfseries}{Chapter\,{\thechapter}}{1em}{}
\titleformat{\chapter}{\centering\Huge\bfseries}{第\,\zhnumber{\thechapter}\,章}{1em}{}    %設定章節樣式

% ----- 中国語,日本語,韓国語使用の設定
\usepackage{xeCJK}
\usepackage{lipsum}
\usepackage{fontspec}
\usepackage{type1ec,type1cm}
\defaultfontfeatures{AutoFakeBold=2.5,AutoFakeSlant=.2}
\setmainfont[Path=\string./fonts/]{times}[        %預設英文字體
	Extension      = .ttf,
	BoldFont       = *bd,
	ItalicFont     = *i,
	BoldItalicFont = *bi
	]
\setsansfont[Path=\string./fonts/]{times} 		 %使用目錄底下指定字體

\ifx \system \primeval
\setCJKmainfont[Path=\string./fonts/]{edukai-4.0}       %教育部楷書
\setCJKsansfont{AR PL UKai TW}
\setCJKmonofont{AR PL UKai TW}
\else
\setCJKmainfont{標楷體}
\setCJKsansfont{標楷體}
\setCJKmonofont{標楷體}
\fi
%----------------------------------------------------------------------------------------


\usepackage[final]{pdfpages}    %PDFの各ページを挿入するためのパッケージです

\usepackage{url}         %URLをリンクとして表示するためのパッケージ
\usepackage[redeflists]{IEEEtrantools}   %複数行にわたった数式を書く場合には,IEEEeqnarray 環境が便利です
\usepackage{zhnumber}
\usepackage{indentfirst}      %首段文章縮排套件
\linespread{1.5} %行距1.5倍

% \usepackage[utf8]{inputenc} % Required for inputting international characters
\usepackage[OT1]{fontenc} % Output font encoding for international characters
\usepackage{palatino} % Use the Palatino font by default
\usepackage{indentfirst} \setlength{\parindent}{2em}
\usepackage{xcolor}
\usepackage{lettrine}  % used for chinese bigger capital
\usepackage{Package/slashbox}
\usepackage{booktabs,amsmath}
\usepackage{algorithm}
\usepackage{enumerate}
\usepackage{multirow}
\usepackage[strict]{changepage}

%----------------------------------------------------------------------------------------
% 參考文獻 Reference package
%----------------------------------------------------------------------------------------
\usepackage[redeflists]{IEEEtrantools}   %複数行にわたった数式を書く場合には,IEEEeqnarray 環境が便利です
\usepackage[hyperref=true,backend=biber,sorting=none,backref=true]{biblatex}
\addbibresource{References/\RefName.bib} % The filename of the bibliography

\frontmatter % Use roman page numbering style (i, ii, iii, iv...) for the pre-content pages
\pagestyle{plain} % Default to the plain heading style until the thesis style is called for the body content
\usepackage{makecell}
\cleardoublepage
\usepackage[autostyle=true]{csquotes} % Required to generate language-dependent quotes in the bibliography
\usepackage{datetime}
\usepackage{ifthen}
% \newwatermark[pages=3-112,fontfamily=bch,color=gray!100,scale=4,xpos=4,ypos=20]{\transparent{0.4}\includegraphics[width=1.25cm]{\watermarkimage}}



