%	PACKAGES AND OTHER DOCUMENT CONFIGURATIONS
\documentclass[oneside,
12pt, % The default document font size, options: 10pt, 11pt, 12pt
english, % ngerman for German
onehalfspacing, % Single line spacing, alternatives: onehalfspacing or doublespacing
]{../Configurations/STUST-MS-Thesis} % The class file specifying the document structure

\setcounter{secnumdepth}{6}
\setcounter{tocdepth}{6}

% 字體設定
\usepackage[BoldFont,SlantFont]{xeCJK}		% xelatex 中文套件設定
\usepackage[LGR,T1]{fontenc} 				% 國際字符的輸出字體編碼
\usepackage{libertine}
\usepackage{lmodern} 
\usepackage{palatino} 						% 默認使用 Palatino 字體
\usepackage{lipsum}							% 輕鬆訪問 Lorem Ipsum 和其他虛擬文本
\usepackage{fontspec}						% XELATE 中的高級字體選擇
\usepackage[english]{babel} 				% 外國語系字母支援
\usepackage{lmodern}						% 大綱格式的拉丁現代字體
\usepackage{anyfontsize}					% 在 LATEX 中選擇任何字體大小
\usepackage{xltxtra}						% XETEX 的 LATEX 用戶的“附加”
\usepackage{type1cm,type1ec} 				% 如果安裝使用可縮放版本的 Computer Modern 或 European Computer Modern (EC) 字體

\usepackage{float}

\xeCJKsetup{PunctStyle=kaiming}
\defaultfontfeatures{AutoFakeBold=2.5,AutoFakeSlant=.2}

% 設定英文字體 (Times New Roman)
\setmainfont[\fontpath]{times}
\setsansfont[\fontpath]{times}

% 設定中文字體 (教育部標準楷體)
\setCJKmainfont[\fontpath]{ukai.ttc}					% 設置正文羅馬族的CJK字體,影響\rmfamily和\textrm 的字體
\setCJKmonofont[\fontpath]{edukai-4.0}					% 設置正文等寬族的CJK字體,影響\ttfamily 和 \texttt 的字體
\setCJKsansfont[\fontpath]{edukai-4.0}					% 設置正文無襯線族的CJK字體,影響\sffamily和\textsf 的字體
\setCJKfamilyfont{edukai-4.0}[\fontpath]{edukai-4.0}    % 教育部楷書
\CJKfontspec[\fontpath]{edukai-4.0}
\defaultCJKfontfeatures{AutoFakeBold=2.5,
						AutoFakeSlant=.2}

\renewcommand*{\rmdefault}{lmr}
\renewcommand*{\sfdefault}{lmss}
\renewcommand*{\ttdefault}{lmtt}
\newcommand*{\edukai}{\CJKfamily{edukai-4.0}}

\newCJKfontfamily\kai{edukai-4.0}[\fontpath,AutoFakeBold]
\newCJKfontfamily\mincho{SawarabiMincho-Regular}[\fontpath,AutoFakeBold]
\newCJKfontfamily\ukai{ukai.ttc}[\fontpath,AutoFakeBold]

% 排版套件
\usepackage{indentfirst} \setlength{\parindent}{2em} 		% 首段文章縮排套件
\usepackage[normalem]{ulem}									% 底線樣式
% \usepackage{microtype}									% 與 xeCJK 套件衝突,先註解

% 解決 XeTeX 中文的斷行問題
\XeTeXlinebreaklocale "zh"
\XeTeXlinebreakskip = 0pt plus 1pt
\XeTeXlinebreakskip = 0pt plus 1pt minus 0.1pt

% 參考文獻 Reference package
\usepackage[redeflists]{IEEEtrantools}   %IEEEeqnarray 環境對於編寫跨多行的公式很有用。
\usepackage{inputenc} % 輸入國際字符時需要
\usepackage[hyperref=true,
			backend=biber,
			sorting=none,
			backref=true,
			style=ieee,
			defernumbers]{biblatex}

\DefineBibliographyStrings{english}{%
			backrefpage = {page},% originally "cited on page"
			backrefpages = {pages},% originally "cited on pages"
			}

\usepackage{makecell}
\usepackage{csquotes} % 需要在參考書目中生成與語言相關的引用

% solve the Underfull \hbox (badness 2205) issues
\usepackage{etoolbox}
% Variant A
\apptocmd{\sloppy}{\hbadness 10000\relax}{}{}	
% Variant B
% \apptocmd{\thebibliography}{\raggedright}{}{}

% 參考文獻檔名連結
\addbibresource{\RefPath} 
\addbibresource{\RefBook}
\pagestyle{plain} % 預設為普通標題樣式,直到正文內容調用論文樣式

%設定超連結文字顏色,目錄連結:黑色,url:藍色,cite:黑色
\usepackage{hyperref} 

\hypersetup{ % 自定義超連結
			pdfpagemode={UseOutlines}, %
			bookmarksopen=true, %
			bookmarksopenlevel=0, %
			bookmarksnumbered=true, %
			% hypertexnames=true, %
			% linktocpage=true, % Set link at page number
			linktoc=all, % Set link at page number and texname
			colorlinks=true, % Set to false to disable coloring links
			citecolor=black, % The color of citations
			linkcolor=black, % The color of references to document elements (sections, figures, etc)
			urlcolor=blue, % The color of hyperlinks (URLs)
			pdfstartview={FitV}, %
			pdftitle = {The title}, % 
			filecolor=magenta, %
			unicode, %
			breaklinks=true %
}

\pdfstringdefDisableCommands{ % 如果節標題(或打印到 pdf 書籤的任何內容)中有明確的換行符,則將其替換為空格
   \let\\\space
}

\usepackage{array}
\usepackage{longtable}
\usepackage{colortbl}			%表格標題註解之巨集套件
\usepackage{pdfpages}    		%插入 PDF 套件
\usepackage{url}         		%將 URL 顯示為超連結的套件
\usepackage{varioref}

\usepackage{zhnumber}
\zhnumsetup{style={Traditional,Normal}}

\linespread{1.5} 				% 行距1.5倍

\usepackage{xparse}
\usepackage{docmute}
\usepackage{datetime}
\usepackage{xcolor}
\usepackage{lettrine}  			% 用於中文更大的標題
\usepackage{../Package/slashbox}
\usepackage{booktabs}
\usepackage{multirow}
\usepackage{forest}
\usepackage{ifthen}
\usepackage[strict]{changepage}
\usepackage{lscape} 			% 橫向放置文檔的選定部分
% \usepackage{atbegshi}			% 文字轉向套件
\usepackage{dcolumn} 			% 對齊表格列中數字的小數點
% \usepackage{blindtext}
\usepackage{enumerate}
\usepackage{enumitem}
\setenumerate[1]{itemsep=0pt,partopsep=0pt,parsep=\parskip,topsep=5pt}
\setitemize[1]{itemsep=0pt,partopsep=0pt,parsep=\parskip,topsep=5pt}
\setdescription{itemsep=0pt,partopsep=0pt,parsep=\parskip,topsep=5pt}

% 繪圖函數
\usepackage{pgfplots}
\usepgfplotslibrary{colorbrewer}
\pgfplotsset{compat=newest,compat/show suggested version=false}

% 圖片套件 Image functions
\usepackage{tikz} % 繪製自定義所需形狀
\usepackage{eso-pic,picture}

% 演算法區塊名稱變更
\renewcommand{\ALG@name}{演算法}

% 程式碼套件
\usepackage{listings}
\lstset{
			language={C},   %語言指定.C言語ならCとします
			basicstyle={\ttfamily}, %標準字體
			identifierstyle={\small},
			commentstyle={\small\ttfamily \color[rgb]{0,0.5,0}},    %註釋字體
			keywordstyle={\small\bfseries \color[rgb]{0,0,1}},      %保留字的字體(int、if 等)
			stringstyle={\small\ttfamily \color[rgb]{1,0,1}},
			frame=tRBl, %指定框架
			framesep=10pt, %框架和內容之間的間距(程式碼)
			breaklines=true, %自動換行
			linewidth=15cm, %框架寬度
			lineskip=-0.5ex, %行距調整
			columns=[l]{fullflexible},  %根據字體調整列寬差異
			numbers=left,
			stepnumber=1,   %行號遞增
			numbersep=14pt,
			tabsize=2, %指定 Tab 應該有多少個字符
			morecomment=[l]{//}
}

% 過濾警告
\usepackage{silence}
\WarningFilter{biblatex}{File 'english-ieee.lbx'}
\WarningsOff[everypage]% Suppress warnings related to package everypage

\usepackage{caption}
% \usepackage{subcaption}
